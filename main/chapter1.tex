%!TEX root = ../mtrgrmetod.tex
\chapter{Основні положення}
Розрахунково-графічна робота (РГР) з дис\-цип\-ліни «Мікропроцесорна техніка» є важливою складовою частиною підготовки технічних фахівців галузей знань 12 -- «Інформаційні технології» та 15 -- «Автоматизація та приладобудування». Написання розрахунково-графічної роботи є обов’язковим етапом у вивченні програмного матеріалу названої дисципліни.

\textit{Метою} виконання розрахунково-графічної роботи є поглиблення набутих теоретичних знань з дисципліни «Мікропроцесорна техніка»; формування практичних навичок алгоритмізації та програмування, вирішення актуальних питань, пов’язаних з проектуванням мікропроцесорних систем; застосування набутого студентами у процесі навчання науково-технічного потенціалу.

У процесі досягнення зазначеної мети вирішуються такі \textit{завдання}:
\begin{itemize}
\item закріпити та поглибити знання з дисципліни «Мікропроцесорна техніка»;
\item систематизувати методичний інструментарій, оволодіти кон\-крет\-ни\-ми комп'ютерними засобами проектування;
\item сформулювати задачу на проектування та обґрунтувати методи й підходи до подальшої формалізації;
\item логічно і послідовно пройти всі етапи розрахунку та розробки;
\item проаналізувати результати, зробити відповідні вис\-нов\-ки.
\end{itemize}

\textit{Мета} і \textit{завдання} в межах виконання РГР визначаються її темою, структурою, специфікою об’єкта, предмета та інформаційною базою для проведення розрахунків.

Виконання студентами розрахунково-графічної роботи сприяє поєднанню в цілісну систему знань із галузі проектування мікропроцесорних засобів, що, в свою чергу, дозволяє їм –- майбутнім інженерам -- сформувати чіткі уявлення про функціонування таких пристроїв та навчитись використовувати їх на практиці.

%Керівник розрахунково-графічної роботи надає допомогу в уточненні змісту, скла\-дан\-ні завдання для виконання курсової роботи. Керівник також сприяє процесу збирання та отримання необхідного матеріалу для написання курсової роботи, рекомендує основну та додаткову літературу, проводить регулярні консультації; розробляє календарний графік виконання етапів роботи та слідкує за його дотриманням, перевіряє роботу, робить відповідні зауваження і вирішує питання про можливість допуску до захисту.

На початку семестру студент отримує індивідуальне завдання на розрахунково-графічну роботу, оформлене відповідно до вимог, що висуваються до такого роду документів. Як правило, індивідуальне завдання -- це аркуш паперу формату А4, на якому в стислій формі подаються вихідні дані для проведення роботи, визначаються числові значення необхідних параметрів та наводиться орієнтовний зміст роботи. Пакет індивідуальних завдань для студентів розглядається на засіданні кафедри і візується завідувачем кафедри на початку семестру. При отриманні індивідуального завдання у відповідній графі бланка студент ставить свій підпис. Свій підпис у відповідній графі ставить і керівник розрахунково-графічної роботи. Варіанти завдань наведені в додатку \ref{apdx:tasks} даних методичних вказівок.

Після отримання індивідуального завдання студент розробляє план роботи, який узгоджує з керівником. На базі розробленого плану формується зміст роботи та перелік необхідних розділів та додатків пояснювальної записки.

Після того, як буде сформовано зміст роботи, студент приступає безпосередньо до виконання робіт, окреслених розробленим планом, та формування пояснювальної записки згідно з сформованим змістом.

Виконання розрахунково-графічної роботи передбачає вивчення літературних джерел і підбір ілюстративного матеріалу. В першу чергу доцільно звертатися до навчальних посібників, які в системному порядку викладають основний зміст курсу. Інформаційною базою для виконання розрахунково-графічної роботи є технічна література у галузі; підручники і навчальні посібники, які в системному порядку викладають основні проблемні та актуальні питання цифрової й мікропроцесорної техніки.

%Особливу увагу слід приділити вивченню змісту основоположних теоретичних і практичних питань моделювання і організації та проведення обчислювального експерименту. При вивченні монографій, журнальних статей, іншої спеціальної літератури з питань, що безпосередньо відносяться до теми курсової роботи, необхідно скласти конспект, викладаючи зміст своїми словами. Такий підхід дозволить забезпечити правильне розуміння вивченого матеріалу, а також дасть можливість самостійно викласти зміст курсової роботи. В якості ілюстративного матеріалу слід підібрати заповнені аналітичні таблиці, графіки, схеми, алгоритми рішення завдань, малюнки, схеми взаємозв'язку показників, і інше.