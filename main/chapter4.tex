%!TEX root = ../ompkrmetod.tex
\chapter{Вимоги до оформлення}
Пояснювальна записка до курсової роботи виконується згідно ДСТУ 3008 – 2015. Мова курсової роботи державна, стиль науковий, чіткий, без орфографічних і синтаксичних помилок; послідовність логічна.

Текст курсової роботи друкується на комп’ютері з одного боку стандартного аркуша односортного паперу формату A4 (210$\times$297 мм). Гарнітура Times New Roman, розмір шрифту 14 пунктів, інтервал 1,5 ($\approx$~28-30 рядків на сторінку). При написанні дотримуються наступних розмірів полів: верхній, лівий і нижній -- не менше 20 мм, правий -- не менше 10 мм. Абзаци в тексті починають відступом, що дорівнює 1,27 см

Під час виконання курсової роботи необхідно дотримуватись рівномірної щільності, контрастності й чіткості зображення. Всі лінії, літери, цифри і знаки повинні бути чіткими та однаково чорними впродовж усієї роботи.

Номера сторінок слід проставляти арабськими цифрами у правому верхньому куті аркушу без крапки в кінці, дотримуючись наскрізної нумерації впродовж усього тексту роботи. Титульний аркуш включають до загальної нумерації сторінок роботи, проте номер сторінки на титульному аркуші не проставляють.

Заголовки структурних частин (розділів) курсової роботи пишуть великими літерами симетрично до тексту, крапка в кінці заголовку не ставиться. Переноси частини слів в заголовку не допускаються, на інший рядок слово переноситься повністю. Якщо заголовок складається з двох речень, то вони розділяються крапкою. Кожний наступний розділ роботи починають з нової сторінки. Розділи нумеруються арабськими цифрами в межах всієї курсової роботи, проте розділам «ЗМІСТ», «ВСТУП», «ВИСНОВКИ», «ПЕРЕЛІК ПОСИЛАНЬ» номера не присвоюють. Крапка після цифри не проставляєься. Заголовки розділів відбивають знизу від основного текту порожнім рядком або інтервалом в 1-1,5 розміру основного шрифту.

Заголовки підрозділів пишуться малими літерами окрім першої і розміщуються з абзацу. Переноси частини слів в підзаголовку не допускаються, на інший рядок слово переноситься повністю. Якщо підзаголовок складається з двох речень, то вони розділяються крапкою. Не допускається розміщувати назву підрозділу, а також пункту й підпункту в нижній частині сторінки, якщо після неї розміщено тільки один рядок тексту. Підрозділи нумерують арабськими цифрами в межах розділу («1.1 Перший підрозділ першого розділу», «2.3 Третій підрозділ другого розділу»), крапку після останньої цифри не проставляють. Заголовки підрозділів відбивають знизу і згори від основного текту порожнім рядком або інтервалом в 1-1,5 розміру основного шрифту.

Формули, що входять до курсової роботи, нумерують в межах розділу. Номер формули складається з номера розділу та порядкового номера формули, розділених крапкою. Номер формули розташовують з правого боку на рівні формули в круглих дужках. Посилання в тексті на номер формули дають в дужках, наприклад, «... за формулою (1.1)».

Пояснення символів та числових коефіцієнтів наводять під формулою. Пояснення кожного символу подається з нового рядка в тій послідовності, в якій символи зустрічаються в формулі. Перший рядок пояснення починається зі слова «де» без двокрапки після нього. 

Формули, що записані одна за одною та не розділені текстом, розділяються комою. Рівняння і формули необхідно виділяти з тексту в окремий рядок. Формули відбивають знизу і згори від основного текту порожнім рядком або інтервалом в 1-1,5 розміру основного шрифту.

Ілюстративні матеріали (таблиці і рисунки) розміщуються в тексті пояснювальної записки до курсової роботи або виносяться в додатки. Ілюстрація має розташовуватись одразу після посилання на неї в тексті, або на наступній сторінці, якщо для розміщення її на поточній сторінці не вистачає місця.

Всі ілюстрації нумеруються арабськими цифрами в межах розділу і повинні мати назву. Номер ілюстрації складається з номера розділу та порядкового номера ілюстрації, розділених крапкою, а назва ілюстрації подається після номеру і відділяється від нього знаком «тире», наприклад, «Рисунок 1.1 -- Схематичне зображення процесу переробки», «Таблиця 1.1 -- Результати комп'ютерного моделювання». Крапка в кінці заголовка ілюстрації не ставиться.

Рисунки підписують знизу симетрично до тексту і відбивають від основного текту порожнім рядком або інтервалом в 1-1,5 розміру основного шрифту.

Таблиці підписують згори вірівнюючи назву по лівому краю таблиці і відбивають від основного текту порожнім рядком або інтервалом в 1-1,5 розміру основного шрифту.

У разі перенесення частини таблиці на інший аркуш (сторінку) слово «Таблиця» та її номер вказують лише один раз -- ліворуч над першою частиною таблиці; над іншими частинами пишуть «Продовження табл.» із зазначенням номера таблиці, наприклад: «Продовження табл. 1.2».

Ілюстративний матеріал може бути оформлений у вигляді додатків. Додатки представляють з себе окремі розділи пояснювальної записки до курсової роботи, що розташовуються після переліку посилань. Як і будь-який розділ додатки мають відображатись в змісті пояснювалної записки і мати наскрізну нумерацію сторінок.

На відміну від звичайних розділів заголовок додатку записують маленькими літерами окрім першої і позначають великими літерами української абетки, починаючи з А, за винятком літер Ґ, Є, З, І, Ї, Й, О, Ч, Ь, наприклад, «Додаток А».  Заголовок додатку розташовують симетрично відносно тексту окремим рядком. Кожний наступний додаток починають з нової сторінки.

Текст кожного додатка, за необхідності, може бути поділений на розділи та підрозділи, пронумеровані в межах кожного додатка: перед кожним номером ставлять позначення додатка (літеру) і крапку, наприклад: «А.2» (другий розділ додатка А). Рисунки, таблиці та формули, розміщені в додатках, нумерують у межах кожного додатка, наприклад: «Рисунок Д.1.2» (другий рисунок першого розділу додатка Д).

При оформленні списку використаної літератури бібліографічний опис складають безпосередньо за друкованим твором або виписують з каталогів і бібліографічних покажчиків повністю без пропусків будь-яких елементів, скорочення назв і т. ін.

Список використаних джерел повинен мати суцільну нумерацію. Використані джерела можна розміщувати в один з таких способів: за абеткою (за першою літерою прізвища автора або першого слова заголовка), у порядку розташування посилань у тексті. Оформлення літературних джерел здійснюється відповідно до ДСТУ ГОСТ 7.1:2006

Оформлена у відповідності до сформульованих вимог та повністю укомплектована курсова робота повинна бути переплетена (зброшурована).

На першій (титульній) сторінці студент повинен поставити свій підпис та дату остаточного завершення роботи.