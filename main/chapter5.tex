%!TEX root = ../ompkrmetod.tex
\chapter{Порядок організації захисту та критерії оцінювання}
Виконана курсова робота у встановлений термін подається на кафедру та після реєстрації передається науковому керівнику для перевірки і підготовки висновку. У висновку відмічаються позитивні сторони та недоліки курсової роботи, оцінюється наявність елементів творчого пошуку та новизни, вказується і обсяг охопленої інформації, дотримання вимог оформлення роботи, робиться висновок щодо допуску до захисту роботи і виставляється попередня оцінка (<<відмінно>>, <<добре>>, <<задовільно>>, «незадовільно»). Якщо курсова робота отримує оцінку «незадовільно», вона повертається студентові для доопрацювання. До переробленої курсової роботи, зданої на повторну перевірку, обов’язково додається попередній висновок.

Захист курсової роботи здійснюється прилюдно перед комісією у складі не менше двох викладачів з обов'язковою присутністю керівника. Термін захисту визначається графіками навчального процесу та затверджується керівником відповідного деканату. Процедура захисту передбачає стислий (до 5 хвилин) виклад студентом основних результатів проведеного дослідження та пропозиції. Після доповіді студент відповідає на всі запитання членів комісії. В процесі захисту можуть використовуватись таблиці, схеми, графіки.

Склад комісії з захисту курсових робіт (не менше двох осіб) призначається завідувачем кафедри.

При аналізі та оцінюванні як самої курсової роботи, так і рівня презентації її результатів увага звертається в першу чергу на:

\begin{itemize}
\item відповідність змісту курсової роботи темі та затвердженому плану;
\item уміння студента визначати найсуттєвіші проблемні питання, що потребують концептуального вирішення;
\item коректність використання понятійного апарату;
\item відповідність логічної побудови роботи поставленим цілям і завданням;
\item обсяг масиву опрацьованої інформації;
\item різноманітність опрацьованих інформаційних джерел;
\item широту й адекватність використаних першоджерел;
\item наявність нестандартних елементів аналізу та діагностики;
\item різноманітність використаних способів порівняння інформації;
\item здатність студента до систематизації вихідної інформації;
\item глибину опрацювання матеріалу;
\item глибину опрацювання проблеми;
\item адекватність запропонованих заходів виявленим проблемам;
\item чіткість визначеної позиції автора;
\item аргументованість, переконливість обґрунтування запропонованих рішень;
\item уміння студента стисло, послідовно та чітко викласти сутність і результати дослідження;
\item розвиненість мови викладення роботи, оригінальність стилю;
\item наявність посилань на джерела, з яких запозичено будь-яку інформацію, та дотримання етики цитування;
\item ступінь самостійності у проведенні дослідження;
\item загальне оформлення дослідження;
\item якість підготовки наочного матеріалу;
\item логічність, конкретність і переконливість доповіді;
\item повноту відповідей на запитання;
\item здатність аргументовано захищати свої пропозиції, думки, погляди;
\item вільне володіння математичною термінологією;
\item загальний рівень підготовки студента.
\end{itemize}

Оцінка вноситься у відомість та залікову книжку студента і перегляду (повторному захисту) не підлягає.

Курсова робота не допускається до захисту і повертається на доопрацювання, якщо:
\begin{itemize}
\item роботу подано на кафедру (на перевірку) для рецензування з порушенням термінів, установлених кафедрою (викладачем, який викладає дану дисципліну);
\item роботу написано на тему, що не внесена до переліку тем курсових робіт з даної дисципліни або не погоджена з викладачем;
\item робота має ознаки плагіату;
\item структура і логіка побудови плану роботи не відповідає вимогам та темі курсової роботи;
\item курсову роботу не зброшуровано (тобто аркуші не скріплені).
\end{itemize}

Необхідні консультації надає викладач кафедри, який перевіряє якість виконання курсової роботи та робить відповідні зауваження.